\documentclass[a4paper,8pt]{article}
\usepackage[utf8]{inputenc}
\usepackage[top=0.75in, bottom=0.75in, left=0.55in, right=0.85in]{geometry}
\usepackage{graphicx}
\usepackage{palatino}
\usepackage{tabularx}
\usepackage{enumitem}
\geometry{textwidth=7cm}
\newlength{\outerbordwidth}
\pagestyle{empty}
\raggedbottom
\raggedright
\usepackage[svgnames]{xcolor}
\usepackage{framed}
\usepackage{tocloft}
\definecolor{mygrey}{gray}{0.75}
\setlength{\outerbordwidth}{3pt}  % Width of border outside of title bars
\setlength{\evensidemargin}{-0.25in}
\setlength{\headheight}{0in}
\setlength{\headsep}{0in}
\setlength{\oddsidemargin}{-0.25in}
\setlength{\paperheight}{11in}
\setlength{\paperwidth}{8.5in}
\setlength{\tabcolsep}{0in}
\setlength{\textheight}{9.5in}
\setlength{\textwidth}{7in}
\setlength{\topmargin}{-0.3in}
\setlength{\topskip}{0in}
\setlength{\voffset}{0.1in}
\setlist{leftmargin=3.28mm}
\renewcommand*\rmdefault{phv}
\newcommand{\resheading}[1]{{\normalsize \colorbox{mygrey}
{\begin{minipage}
{1\textwidth}{\textbf{\sffamily{\mbox{~}\makebox[5.9in][l]{\large #1} \vphantom{p\^{E}}}}}
\end{minipage}}}}


\newgeometry{left=0.5in,right=0.5in,top=0.5in,bottom=1in}
%%%%%%%%%%%%%%%%%%%%%%%%%%%%%%%%%%%%%%%%%%%%%%%%%%%%%%%%%%%%%%%%%%%%%%
%%%%%%%%%%%%%%%%%%%%        TITLE          %%%%%%%%%%%%%%%%%%%%%%%%%%%
\begin{document}
\begin{tabular*}{7.25in}{l@{\extracolsep{\fill}}r}
{\Large R\large OMIL \Large K\large ADIA} & +91-942-912-0451 \\
{\large D\normalsize EPARTMENT OF \large M\normalsize ECHANICAL \large E\normalsize NGINEERING}& \textit{romil@iitk.ac.in} \\
\large I\normalsize NDIAN \large I\normalsize NSTITUTE OF \large T\normalsize ECHNOLOGY, \large K\normalsize ANPUR & \textit{romil.kadia042@gmail.com} \\
\end{tabular*}
\\

%%%%%%%%%%%%%%%%%%%%%%%%%%%%%%%%%%%%%%%%%%%%%%%%%%%%%%%%%%%%%%%%%%%
%%%%%%%%%%%%%%%%%%       TABULAR          %%%%%%%%%%%%%%%%%%%%%%%%%%%
\resheading{\large A\normalsize CADEMIC \large D\normalsize ETAILS}
 \indent \begin{tabular}{ l @{\hskip 0.365in} l @{\hskip 0.365in} l @{\hskip 0.365in} l @{\hskip 0.365in} l }
\hline
\textbf{Degree} & \textbf{University/Board} & \textbf{Institute/School} & \textbf{Year} & \textbf{CPI/\%} \\
\hline
Master of Technology* & Indian Institute of Technology Kanpur & IIT Kanpur & 2018 & 7.75/10\\
Bachelor of Engineering** & Gujarat Technological University & LDRP-ITR & 2015 & 9.02/10\\
Intermediate/+2 & Gujarat State Board & Amrut, Ahmedabad & 2011 & 84\%\\
Matriculation & Gujarat State Board & Amrut, Ahmedabad & 2009 & 76.5\%\\
\hline
\end{tabular}
\begin{tabular*}{7in}{l@{\extracolsep{\fill}}r}
\small{\textit{*specialization in Solid Mechanics and Design}} & \small{\textit{**Mechanical Engineering}}
\end{tabular*}
%%%%%%%%%%%%%%%%%%%%%%%%%%%%%%%%%%%%%%%%%%%%%%%%%%%%%%%%%%%%%%%%%%
%%%%%%%%%%%%%%%%%% KEY SCHOLATIC ACHIVEMENTS %%%%%%%%%%%%%%%%%%%%%
%%%%%%%%%%%%%%%%%%%%%%%%%%%%%%%%%%%%%%%%%%%%%%%%%%%%%%%%%%%%%%%%%%
\resheading{\large K\normalsize EY \large S\normalsize CHOLASTIC \large A\normalsize CHIVEMENTS}
\begin{itemize}[topsep=0pt]
\setlength{\itemsep}{-3pt}
\item \makebox[\linewidth][s]{Ranked \textbf{1$^{st}$} in \textbf{Bachelor of Engineering} in Department of Mechanical Engineering at LDRP-ITR awarded} with \textbf{Institute Gold Medal} by Kadva Patidal Kelavni Mandal for the same in 2015.
\item Secured \textbf{All India Rank 786} in \textbf{GATE, 2016} among 0.21 million candidates across country.
\item Ranked \textbf{27$^{th}$} in Bachelor of Engineering in Mechanical engineering at Gujarat Technological University.
\item Awarded with \textbf{Maneklal M. Patel Memorial Scholarship} \textit{(given to top 0.1\%)} for \textit{excellent performance} at Kadi Sarva Vishwavidyalaya by President Vallabhbhai M. Patel for academic year 2014-15.
\end{itemize}
%%%%%%%%%%%%%%%%%%%%%%%%%%%%%%%%%%%%%%%%%%%%%%%%%%%%%%%%%%%%%%%%%%%
%%%%%%%%%%%%%%%%% THESIS AND PROJECT %%%%%%%%%%%%%%%%%%%%%%%%%%%%%%
%%%%%%%%%%%%%%%%%%%%%%%%%%%%%%%%%%%%%%%%%%%%%%%%%%%%%%%%%%%%%%%%%%%
\resheading{\large T\normalsize HESIS AND \large P\normalsize ROJECTS}
\begin{tabular*}{7.25in}{l@{\extracolsep{\fill}}r}
%%%%%%%%%%%%%%%%%%%%%%%%%%%%%%%%%%%%%%%%%%%%%%%%%%%%%%%%%%%%%%%%%%%%
\textbf{Dislocation and Disclination Motion Study of Graphene at Zero K} & \textbf{M.Tech Thesis, IIT Kanpur}\\
\textit{MD Simulation}, Thesis Supervisors Dr. Anurag Gupta and Dr. Sakti Singh Gupta & \textit{Dec'16-present}
\end{tabular*}
\begin{itemize}[topsep=0pt]
\setlength{\itemsep}{-3pt}
\item Analysing behaviour of dislocations through the energy variation before and after energy minimization at zero K using \textbf{Molecular Simulations(Tinker)} in Graphene sheets, Carbon Nano Tubes and Fullerene.
\item Dislocations are introduced and input files are generated using \textbf{Python} programming language in Graphene sheet and Carbon Nano Tubes which is modelled as sputtering of carbon atom due to thermal waves.
\item Visualization of energy variation is done in \textbf{MatLab} using all the possible position of Dislocation.
\item Analysing Buckling and its energy variation and  concluded that dislocations are stable at center.
\end{itemize}
\rule{\textwidth}{0.75pt}
%%%%%%%%%%%%%%%%%%%%%%%%%%%%%%%%%%%%%%%%%%%%%%%%%%%%%%%%%%%%%%%%%%%
\begin{tabular*}{7.25in}{l@{\extracolsep{\fill}}r}
\textbf{Design and Development of Centrifugal Type Positive Frictional Clutch} & \textbf{B.E. Project, LDRP-ITR}\\
\textit{Automotive Engineering}, Project Supervisor Prof. D. H. Pandya & \textit{May'14-May'15}
\end{tabular*}
\begin{itemize}[topsep=0pt]
\setlength{\itemsep}{-3pt}
\item Clutch slip phenomena avoided by using the combination of centrifugal, positive and frictional disc clutch.
\item In the first phase modelling of complete system was made using \textbf{Solidworks} and analysis of each component and Assembly as well as sub-Assembly is done using \textbf{ANSYS Static Structure toolbar}.
\item Initial design was over-safe and was intended due to manufacturing constrains of model.
\item Patent has been filed and communication is going on with \textbf{Indian Patent Office} for design related issues.
\item Clutch plate wear decreased to a huge extend and cheaper materials can be employed as current clutch lining material Asbestos-Ferodo is very costly and has negative environmental impact.
\end{itemize}
\rule{\textwidth}{0.75pt}
%%%%%%%%%%%%%%%%%%%%%%%%%%%%%%%%%%%%%%%%%%%%%%%%%%%%%%%%%%%%%%%%%%%%
\begin{tabular*}{7.25in}{l@{\extracolsep{\fill}}r}
\textbf{Analysis of Solid with Elasto-Plastic behaviour} & \textbf{Course Project IIT Kanpur}\\
\textit{Non-Linear FEM}, under guidance Prof. Sumit Basu & \textit{Jan'17-April'17}
\end{tabular*}
\begin{itemize}[topsep=0pt]
\setlength{\itemsep}{-3pt}
\item \textbf{UMAT} and Consistant Tangent stiffness matrix was developed for \textbf{ABAQUS} based on Von-Misces yield condition.
\end{itemize}
\rule{\textwidth}{0.75pt}
%%%%%%%%%%%%%%%%%%%%%%%%%%%%%%%%%%%%%%%%%%%%%%%%%%%%%%%%%%%%%%%%%
\begin{tabular*}{7.25in}{l@{\extracolsep{\fill}}r}
\textbf{Analysis of Neo-Hookean material} &  \textbf{Course Project IIT Kanpur}\\
\textit{Non-Linear FEM}, under guidance Prof. Sumit Basu & \textit{Jan'17-April'17}
\end{tabular*}
\begin{itemize}[topsep=0pt]
\setlength{\itemsep}{-3pt}
\item \textbf{UMAT} and Tangent stiffness matrix was developed for \textbf{ABAQUS} based on Neo-Hookean free energy function.
\end{itemize}
\rule{\textwidth}{0.75pt}
%%%%%%%%%%%%%%%%%%%%%%%%%%%%%%%%%%%%%%%%%%%%%%%%%%%%%%%%%%%%%%%%%
\begin{tabular*}{7.25in}{l@{\extracolsep{\fill}}r}
\textbf{Longitudinal vibration of rod for Linear FEM} & \textbf{Course Project IIT Kanpur}\\
\textit{FEM}, under guidance Prof. P. M. Dixit & \textit{Aug'16-Nov'16}
\end{tabular*}
\begin{itemize}[topsep=0pt]
\setlength{\itemsep}{-3pt}
\item FEM code is developed in \textbf{MatLab} with Lagrangian $C^{2}$ continuous element with variation in number of elements from 20 to 640.
Rod of uniformly varying in cross section area was used with quadratic variation.
\end{itemize}
\rule{\textwidth}{0.75pt}
%%%%%%%%%%%%%%%%%%%%%%%%%%%%%%%%%%%%%%%%%%%%%%%%%%%%%%%%%%%%%%%%
\begin{tabular*}{7.25in}{l@{\extracolsep{\fill}}r}
\textbf{Heat Conduction from 2D plate in Linear FEM} & \textbf{Course Project IIT Kanpur}\\
\textit{FEM}, under guidance Prof. P. M. Dixit & \textit{Aug'16-Nov'16}
\end{tabular*}
\begin{itemize}[topsep=0pt]
\setlength{\itemsep}{-3pt}
\item FEM code is developed in \textbf{MatLab} with Lagrangian $C^{2}$ continuous triangular element.
\end{itemize}
\rule{\textwidth}{0.75pt}
%%%%%%%%%%%%%%%%%%%%%%%%%%%%%%%%%%%%%%%%%%%%%%%%%%%%%%%%%%%%%%%%
\begin{tabular*}{7.25in}{l@{\extracolsep{\fill}}r}
\textbf{Design of Scheme interpreter} & \textbf{Course Project, UC Berkeley(online)}\\
%\textit{FEM}, under guidance Prof. P. M. Dixit & \textit{Aug'16-Nov'16}
\end{tabular*}
\begin{itemize}[topsep=0pt]
\setlength{\itemsep}{-3pt}
\item Scheme interpreter was designed using Python as a part of course project of CS61A.
\end{itemize}
%%%%%%%%%%%%%%%%%%%%%%%%%%%%%%%%%%%%%%%%%%%%%%%%%%%%%%%%%%%%%%%%%%%
%%%%%%%%%%%%%%%%% COMPUTER SKILLS %%%%%%%%%%%%%%%%%%%%%%%%%%%%%%
%%%%%%%%%%%%%%%%%%%%%%%%%%%%%%%%%%%%%%%%%%%%%%%%%%%%%%%%%%%%%%%%%%%
\resheading{\large C\normalsize OMPUTER \large S\normalsize KILLS}
\begin{itemize}[topsep=0pt]
\setlength{\itemsep}{-3pt}
\item \textbf{Programming Language:} Python, C/C++, JAVA, HTML/CSS, Fortran-95(Basic), Scheme.
\item \textbf{Software:} Solidworks, ANSYS, ABAQUS, MatLab, LATEX, Tinker, AutoCAD, Creo-Parametric.
\end{itemize}
%%%%%%%%%%%%%%%%%%%%%%%%%%%%%%%%%%%%%%%%%%%%%%%%%%%%%%%%%%%%%%%%%%
%%%%%%%%%%%%%%%%%% COURSES UNDERTAKEN %%%%%%%%%%%%%%%%%%%%%%%%%%%%
%%%%%%%%%%%%%%%%%%%%%%%%%%%%%%%%%%%%%%%%%%%%%%%%%%%%%%%%%%%%%%%%%%
\resheading{\large C\normalsize OURSES \large U\normalsize NDERTAKEN}
\begin{itemize}[topsep=0pt]
\setlength{\itemsep}{-3pt}
\item \textbf{Mechanical Engineering:} Strength of Material, Solid Mechanics, Molecular Dynamics Simulations, Finite Element Method (Linear, Non-linear), Vibration of Continuous Systems (1D,2D), Advance Dynamics.
\pagebreak
\item \textbf{Computer Science and Engineering:} Machine Learning(CS771A, IITK(Audit)), Data Structure and Algorithm in JAVA(CS61B,UC Berkeley(online)), Structure and Interpretation of Computer Program(CS61B, UC Berkeley(online)), Introduction to Algorithms(6.006, MIT OCW), Computer system Engineering(6.033, MIT OCW), Operating System(CS140, Stanford(online)).
\end{itemize}
%%%%%%%%%%%%%%%%%%%%%%%%%%%%%%%%%%%%%%%%%%%%%%%%%%%%%%%%%%%%%%%%%%%
%%%%%%%%%%%%%%%%% POR %%%%%%%%%%%%%%%%%%%%%%%%%%%%%%
%%%%%%%%%%%%%%%%%%%%%%%%%%%%%%%%%%%%%%%%%%%%%%%%%%%%%%%%%%%%%%%%%%%
\resheading{\large P\normalsize OSITION \large O\normalsize F \large R\normalsize ESPONSIBILITY }
\begin{tabular*}{7.25in}{l@{\extracolsep{\fill}}r}
\textbf{Department Placement Co-ordinator(ME)} & \textbf{IIT Kanpur}\\
Student Placement Office & \textit{May'17-Present}
\end{tabular*}
\begin{itemize}[topsep=0pt]
\setlength{\itemsep}{-3pt}
\item Integral member of \textbf{4-tier team} of \textbf{120 members} to facilitate placements of \textbf{1200+} graduating students.
\item Developing and strengthening contact with new core companies and inviting them for upcoming placement session.% & internship
\item Responsible for guiding and helping the mechanical engineering students in their placement preparation.
\end{itemize}
\rule{\textwidth}{0.75pt}
%%%%%%%%%%%%%%%%%%%%%%%%%%%%%%%%%%%%%%%%%%%%%%%%%%%%%%%%
\begin{tabular*}{7.25in}{l@{\extracolsep{\fill}}r}
\textbf{Web Manager} & \textbf{IIT Kanpur}\\
Association of Mechanical Engineers & \textit{July'17-Present}
\end{tabular*}
\begin{itemize}[topsep=0pt]
\setlength{\itemsep}{-3pt}
\item Led a team of \textbf{4 people} in planning, maintaining and improving the online presence of IIT Kanpur’s student body of Department of Mechanical Engineering functionally.
\item Started a web platform (\textbf{AME Digital Library}) to enable collaboration of academic literature among students.
\end{itemize}
\rule{\textwidth}{0.75pt}
%%%%%%%%%%%%%%%%%%%%%%%%%%%%%%%%%%%%%%%%%%%%%%%%%%
\begin{tabular*}{7.25in}{l@{\extracolsep{\fill}}r}
\textbf{Teaching Assistant} & \textbf{IIT Kanpur}\\
Mathematics for Engineers(ME681A) & \textit{Jan'17-May'17}
\end{tabular*}
\begin{itemize}[topsep=0pt]
\setlength{\itemsep}{-3pt}
\item Graded assignments and clarifyed queries of \textbf{55} students of $1^{st}$ year M.Tech in mechanical engineering.
\end{itemize}
\rule{\textwidth}{0.75pt}
%%%%%%%%%%%%%%%%%%%%%%%%%%%%%%%%%%%%%%%%%%%%%%%%%%
\begin{tabular*}{7.25in}{l@{\extracolsep{\fill}}r}
\textbf{Teaching Assistant} & \textbf{IIT Kanpur}\\
Technical Art(TA101) & \textit{Aug'17-Present}
\end{tabular*}
\begin{itemize}[topsep=0pt]
\setlength{\itemsep}{-3pt}
\item Graded assignments, Drawing sheets and clarifyed queries of \textbf{450+} students of $1^{st}$ year B.Tech from all respective branches. 
\end{itemize}
%%%%%%%%%%%%%%%%%%%%%%%%%%%%%%%%%%%%%%%%%%%%%%%%%%%%
%%%%%%%%%%%%%%%%% ECA %%%%%%%%%%%%%%%%%%%%%%%%%%%%%%
%%%%%%%%%%%%%%%%%%%%%%%%%%%%%%%%%%%%%%%%%%%%%%%%%%%%
\resheading{\large E\normalsize XTRA-CURRICULAR \large A\normalsize CTIVITIES}
\begin{itemize}[topsep=0pt]
\setlength{\itemsep}{-3pt}
\item Secured \textbf{3$^{rd}$} position in \textbf{Technical Quizz} among  \textbf{480+} students  at Mad-Labs'12(annular deparmental techfest) at 
\textbf{LDRP-ITR} and awarded with \textbf{Bronze medal} for the same.
\item Secured \textbf{2591$^{th}$} rank \textbf{SNACK Down'17} a competitive programming challange held annually with \textbf{22000}+ candidates globally by \textbf{CodeChef}.
\item \textbf{Presented review paper} at Mad-Labs'12 and Xenesis'13(annular departmental techfest) at \textbf{LDRP-ITR}.
\item Attended automobile workshop at Xenesis'13.
\end{itemize}
\end{document}